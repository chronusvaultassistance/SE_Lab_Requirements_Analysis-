\documentclass{report}
\usepackage[margin=1in]{geometry}
\usepackage{hyperref}
\usepackage{tocloft} % For table of contents customization
\usepackage{enumitem} % For better lists

\title{Requirement Analysis Introduction Report \\ University Portal Management System}
\author{Raihan and Team}
\date{ }

\begin{document}

\maketitle

\tableofcontents % Indexing via Table of Contents

\chapter{Introduction}
The requirement analysis report provides a comprehensive introduction to the requirements for the University Portal Management System. The system is designed as a secure, role-based web application to facilitate interactions among students, teachers, and administrators in a university setting. It addresses key challenges in academic management, such as manual processes, communication gaps, and lack of centralized resources. The following sections outline a brief summary of the project, user interface overview, functional requirements, project boundaries, team roles, development process model, and technical processes. This analysis ensures that the system meets user needs while adhering to modern software engineering practices.

\section{Brief Summary of the University Portal Management System}
The University Portal Management System is a web-based platform aimed at streamlining academic operations within a university environment. It supports three primary user entities: students, teachers (with sub-roles including course teachers, advisors, and department chairmen), and administrators. The system enables efficient management of course enrollments, lab report submissions, internal communications, attendance tracking, and administrative oversight.

Key problems addressed by the system include:
- \textbf{Manual Enrollment and Approvals}: Traditional methods often lead to delays, paperwork errors, and inefficiencies. The portal automates enrollment requests, allowing students to submit applications semester-wise and section-specific, with advisors approving them digitally.
- \textbf{Scattered Communication}: Students and faculty rely on disparate tools like emails or messaging apps, leading to missed updates. The system introduces an internal mailing feature where students can send messages to specific roles (e.g., course teachers or advisors), with restrictions to prevent abuse, such as limiting leave applications to five per month.
- \textbf{Attendance Management Issues}: Manual attendance tracking is prone to inaccuracies and fails to enforce policies consistently. The portal allows teachers to upload attendance data, automatically designating students with less than 50\% attendance as "Fail" and notifying them, while permitting updates for corrections.
- \textbf{Lack of Centralized Resources}: Course materials, lesson plans, and submissions are often decentralized. The system provides a unified repository for uploads and views, supporting various file formats (PDF, DOCX, LaTeX, PPT).
- \textbf{Administrative Overload}: Administrators handle everything manually, from role assignments to password management. The portal grants admins full control, including viewing decrypted student passwords for recovery purposes (with security caveats).

The motivation for this project stems from the need to digitize academic workflows, reduce administrative burdens, enhance transparency, and improve user satisfaction. By automating routine tasks, the system allows faculty to focus on teaching and students on learning, ultimately fostering a more efficient educational ecosystem.

\section{Summary of the User Interface}
The user interface (UI) of the University Portal Management System is designed to be intuitive, responsive, and role-specific, ensuring that each user type accesses only relevant features. Built with React.js for the frontend, the UI emphasizes simplicity, accessibility, and mobile compatibility. Dashboards are customized: students see enrollment options and submission forms; teachers have tools for uploads and approvals; admins have comprehensive controls.

The UI incorporates modern design principles, such as clean navigation menus, real-time notifications for updates (e.g., enrollment status or low attendance alerts), and secure forms for data entry. Visual elements include progress trackers for attendance and interactive lists for course selections. This summary highlights how the UI supports functional requirements while maintaining usability.

\subsection{Functional Requirements and Deliverables}
The functional requirements define what the system must do to meet user needs. These are categorized by user roles and include detailed specifications for deliverables.

\textbf{Student Functional Requirements:}
- Registration: Students can create accounts with auto-generated 12-digit unique IDs (last four digits batch-specific), including details like name, department, age, batch, semester, and section.
- Authentication: Secure login/logout with password protection.
- Enrollment: Request enrollment in current semester courses, view status, and receive notifications.
- Submissions: Upload lab reports in supported formats (PDF, DOCX, LaTeX, PPT) to enrolled courses.
- Communication: Send internal mails to course teachers, advisors, or chairmen, with a limit of five leave applications per month to advisors.
- Viewing: Access lesson plans and course materials for enrolled courses.
- Deliverables: Student dashboard with forms, upload interfaces, and notification panels.

\textbf{Teacher Functional Requirements:}
- Role-Specific Access: Course teachers upload materials/lesson plans and view mails from enrolled students; advisors approve enrollments specific batch and handle batch-specific mails; chairmen assign roles and view department-wide communications.
- Attendance Management: Upload attendance percentages; system auto-fails students below 50\% and sends notifications; allow updates to override.
- Communication: Reply to student mails with threading support.
- Deliverables: Teacher dashboard with upload forms, approval buttons, and filtered inboxes.

\textbf{Admin Functional Requirements:}
- Oversight: Manage all entities, create courses, assign teachers/advisors, and view all records.
- Security: Access decrypted student passwords for administrative purposes.
- Deliverables: Admin console with search, edit, and reporting tools.

Non-functional requirements include performance (response time < 2 seconds), security (role-based access control, data encryption), and scalability (handle 1000+ users). Deliverables encompass the complete source code, database schema, user manuals, and deployment guides.

\subsection{Project Boundaries}
The project boundaries define the scope to ensure focused development and avoid scope creep.

\textbf{Inclusions:}
- Core features: Authentication, enrollment, submissions, mailing, attendance tracking with auto-fail, and role assignments.
- Supported Formats: File uploads limited to PDF, DOCX, LaTeX, PPT.
- Internal Communications: Database-stored mails only; no external email integration in initial phase.
- User Roles: Limited to students, teachers (with sub-roles), and admins.

\textbf{Exclusions:}
- Advanced Analytics: No built-in reporting dashboards for trends (e.g., attendance statistics across semesters).
- Mobile App: Web-based only; no native iOS/Android apps.
- Payment Integration: No handling of fees or financial transactions.
- External Integrations: No API connections to third-party systems like LMS or HR software.
- Multilingual Support: English-only interface.
- Assumptions: Users have basic computer literacy; system assumes stable internet access.

These boundaries help prioritize essential features while allowing for future expansions.

\subsection{Team Roles and Responsibilities}
The development team consists of four members, each with defined roles to ensure efficient collaboration.

- \textbf{Raihan (Team Lead \& Full-Stack Developer)}: Oversees project architecture, implements backend APIs (Node.js, Express), handles authentication and core features like enrollment and mailing. Responsible for integration and final reviews.
- \textbf{ Raihan Zaman (Frontend Devloper)}: Develops React.js components, designs responsive dashboards, forms, and user flows. Ensures UI/UX compliance with accessibility standards.
- \textbf{Mostafa Wasif (Database \& Backend Developer)}: Designs MySQL schema, implements Sequelize ORM models, manages file uploads (Multer), and develops attendance logic including auto-fail mechanisms.
- \textbf{Member 3 (Features \& Testing Specialist)}: Implements specific modules like messaging, submissions, and attendance; conducts unit/integration testing, bug fixing, and prepares documentation.

All members contribute to code reviews, sharing codes resources and version Control via git and hub, meetings, and the final presentation. This distribution leverages individual strengths for balanced progress.

\subsection{Development Process Model}
The project adopts an \textbf{Agile Development Process Model} with Scrum elements to allow flexibility and iterative improvements. This model is chosen over Waterfall due to the evolving nature of requirements in educational software as not all requirements are fixed yet.

- \textbf{Sprints}: 2-week cycles focusing on specific features (e.g., Sprint 1: Authentication; Sprint 2: Enrollment).
- \textbf{Daily Stand-ups}: Short meetings for progress updates.
- \textbf{Backlog Management}: Prioritized features using tools like Jira.
- \textbf{Reviews \& Retrospectives}: End-of-sprint demos and feedback sessions.
- \textbf{Advantages}: Accommodates changes, such as adding attendance updates, and ensures continuous user feedback.

This model promotes collaboration, quick iterations, and high-quality deliverables within the timeline.

\subsection{Technical Process}
The technical process outlines the methodologies and tools for implementation, ensuring robustness and maintainability.

- \textbf{Frontend Development}: Use React.js with components for modularity (e.g., LoginComponent, DashboardComponent). State management via Redux for complex interactions like real-time notifications.
- \textbf{Backend Development}: Node.js with Express for RESTful APIs. ORM (Sequelize) for MySQL interactions to handle CRUD operations securely. OOP concepts: Base User class with inheritance for Student/Teacher/Admin, polymorphism for role-specific methods.
- \textbf{Database Management}: MySQL via XAMPP; schema includes tables for users, courses, enrollments, mails, etc. Triggers for auto-ID generation and attendance checks.
- \textbf{Security Processes}: Bcrypt for password hashing (reversible encryption for students); JWT for authentication; validation for file uploads and mail limits.
- \textbf{Testing}: Unit tests (Jest), integration tests, and manual UI testing.
- \textbf{Deployment}: Local setup with XAMPP; potential cloud migration (e.g., AWS).

This process emphasizes best practices like version control (Git) and code reviews to mitigate risks and ensure a scalable system.

\end{document}